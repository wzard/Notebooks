
% Default to the notebook output style

    


% Inherit from the specified cell style.




    
\documentclass[11pt]{article}

    
    
    \usepackage[T1]{fontenc}
    % Nicer default font (+ math font) than Computer Modern for most use cases
    \usepackage{mathpazo}

    % Basic figure setup, for now with no caption control since it's done
    % automatically by Pandoc (which extracts ![](path) syntax from Markdown).
    \usepackage{graphicx}
    % We will generate all images so they have a width \maxwidth. This means
    % that they will get their normal width if they fit onto the page, but
    % are scaled down if they would overflow the margins.
    \makeatletter
    \def\maxwidth{\ifdim\Gin@nat@width>\linewidth\linewidth
    \else\Gin@nat@width\fi}
    \makeatother
    \let\Oldincludegraphics\includegraphics
    % Set max figure width to be 80% of text width, for now hardcoded.
    \renewcommand{\includegraphics}[1]{\Oldincludegraphics[width=.8\maxwidth]{#1}}
    % Ensure that by default, figures have no caption (until we provide a
    % proper Figure object with a Caption API and a way to capture that
    % in the conversion process - todo).
    \usepackage{caption}
    \DeclareCaptionLabelFormat{nolabel}{}
    \captionsetup{labelformat=nolabel}

    \usepackage{adjustbox} % Used to constrain images to a maximum size 
    \usepackage{xcolor} % Allow colors to be defined
    \usepackage{enumerate} % Needed for markdown enumerations to work
    \usepackage{geometry} % Used to adjust the document margins
    \usepackage{amsmath} % Equations
    \usepackage{amssymb} % Equations
    \usepackage{textcomp} % defines textquotesingle
    % Hack from http://tex.stackexchange.com/a/47451/13684:
    \AtBeginDocument{%
        \def\PYZsq{\textquotesingle}% Upright quotes in Pygmentized code
    }
    \usepackage{upquote} % Upright quotes for verbatim code
    \usepackage{eurosym} % defines \euro
    \usepackage[mathletters]{ucs} % Extended unicode (utf-8) support
    \usepackage[utf8x]{inputenc} % Allow utf-8 characters in the tex document
    \usepackage{fancyvrb} % verbatim replacement that allows latex
    \usepackage{grffile} % extends the file name processing of package graphics 
                         % to support a larger range 
    % The hyperref package gives us a pdf with properly built
    % internal navigation ('pdf bookmarks' for the table of contents,
    % internal cross-reference links, web links for URLs, etc.)
    \usepackage{hyperref}
    \usepackage{longtable} % longtable support required by pandoc >1.10
    \usepackage{booktabs}  % table support for pandoc > 1.12.2
    \usepackage[inline]{enumitem} % IRkernel/repr support (it uses the enumerate* environment)
    \usepackage[normalem]{ulem} % ulem is needed to support strikethroughs (\sout)
                                % normalem makes italics be italics, not underlines
    

    
    
    % Colors for the hyperref package
    \definecolor{urlcolor}{rgb}{0,.145,.698}
    \definecolor{linkcolor}{rgb}{.71,0.21,0.01}
    \definecolor{citecolor}{rgb}{.12,.54,.11}

    % ANSI colors
    \definecolor{ansi-black}{HTML}{3E424D}
    \definecolor{ansi-black-intense}{HTML}{282C36}
    \definecolor{ansi-red}{HTML}{E75C58}
    \definecolor{ansi-red-intense}{HTML}{B22B31}
    \definecolor{ansi-green}{HTML}{00A250}
    \definecolor{ansi-green-intense}{HTML}{007427}
    \definecolor{ansi-yellow}{HTML}{DDB62B}
    \definecolor{ansi-yellow-intense}{HTML}{B27D12}
    \definecolor{ansi-blue}{HTML}{208FFB}
    \definecolor{ansi-blue-intense}{HTML}{0065CA}
    \definecolor{ansi-magenta}{HTML}{D160C4}
    \definecolor{ansi-magenta-intense}{HTML}{A03196}
    \definecolor{ansi-cyan}{HTML}{60C6C8}
    \definecolor{ansi-cyan-intense}{HTML}{258F8F}
    \definecolor{ansi-white}{HTML}{C5C1B4}
    \definecolor{ansi-white-intense}{HTML}{A1A6B2}

    % commands and environments needed by pandoc snippets
    % extracted from the output of `pandoc -s`
    \providecommand{\tightlist}{%
      \setlength{\itemsep}{0pt}\setlength{\parskip}{0pt}}
    \DefineVerbatimEnvironment{Highlighting}{Verbatim}{commandchars=\\\{\}}
    % Add ',fontsize=\small' for more characters per line
    \newenvironment{Shaded}{}{}
    \newcommand{\KeywordTok}[1]{\textcolor[rgb]{0.00,0.44,0.13}{\textbf{{#1}}}}
    \newcommand{\DataTypeTok}[1]{\textcolor[rgb]{0.56,0.13,0.00}{{#1}}}
    \newcommand{\DecValTok}[1]{\textcolor[rgb]{0.25,0.63,0.44}{{#1}}}
    \newcommand{\BaseNTok}[1]{\textcolor[rgb]{0.25,0.63,0.44}{{#1}}}
    \newcommand{\FloatTok}[1]{\textcolor[rgb]{0.25,0.63,0.44}{{#1}}}
    \newcommand{\CharTok}[1]{\textcolor[rgb]{0.25,0.44,0.63}{{#1}}}
    \newcommand{\StringTok}[1]{\textcolor[rgb]{0.25,0.44,0.63}{{#1}}}
    \newcommand{\CommentTok}[1]{\textcolor[rgb]{0.38,0.63,0.69}{\textit{{#1}}}}
    \newcommand{\OtherTok}[1]{\textcolor[rgb]{0.00,0.44,0.13}{{#1}}}
    \newcommand{\AlertTok}[1]{\textcolor[rgb]{1.00,0.00,0.00}{\textbf{{#1}}}}
    \newcommand{\FunctionTok}[1]{\textcolor[rgb]{0.02,0.16,0.49}{{#1}}}
    \newcommand{\RegionMarkerTok}[1]{{#1}}
    \newcommand{\ErrorTok}[1]{\textcolor[rgb]{1.00,0.00,0.00}{\textbf{{#1}}}}
    \newcommand{\NormalTok}[1]{{#1}}
    
    % Additional commands for more recent versions of Pandoc
    \newcommand{\ConstantTok}[1]{\textcolor[rgb]{0.53,0.00,0.00}{{#1}}}
    \newcommand{\SpecialCharTok}[1]{\textcolor[rgb]{0.25,0.44,0.63}{{#1}}}
    \newcommand{\VerbatimStringTok}[1]{\textcolor[rgb]{0.25,0.44,0.63}{{#1}}}
    \newcommand{\SpecialStringTok}[1]{\textcolor[rgb]{0.73,0.40,0.53}{{#1}}}
    \newcommand{\ImportTok}[1]{{#1}}
    \newcommand{\DocumentationTok}[1]{\textcolor[rgb]{0.73,0.13,0.13}{\textit{{#1}}}}
    \newcommand{\AnnotationTok}[1]{\textcolor[rgb]{0.38,0.63,0.69}{\textbf{\textit{{#1}}}}}
    \newcommand{\CommentVarTok}[1]{\textcolor[rgb]{0.38,0.63,0.69}{\textbf{\textit{{#1}}}}}
    \newcommand{\VariableTok}[1]{\textcolor[rgb]{0.10,0.09,0.49}{{#1}}}
    \newcommand{\ControlFlowTok}[1]{\textcolor[rgb]{0.00,0.44,0.13}{\textbf{{#1}}}}
    \newcommand{\OperatorTok}[1]{\textcolor[rgb]{0.40,0.40,0.40}{{#1}}}
    \newcommand{\BuiltInTok}[1]{{#1}}
    \newcommand{\ExtensionTok}[1]{{#1}}
    \newcommand{\PreprocessorTok}[1]{\textcolor[rgb]{0.74,0.48,0.00}{{#1}}}
    \newcommand{\AttributeTok}[1]{\textcolor[rgb]{0.49,0.56,0.16}{{#1}}}
    \newcommand{\InformationTok}[1]{\textcolor[rgb]{0.38,0.63,0.69}{\textbf{\textit{{#1}}}}}
    \newcommand{\WarningTok}[1]{\textcolor[rgb]{0.38,0.63,0.69}{\textbf{\textit{{#1}}}}}
    
    
    % Define a nice break command that doesn't care if a line doesn't already
    % exist.
    \def\br{\hspace*{\fill} \\* }
    % Math Jax compatability definitions
    \def\gt{>}
    \def\lt{<}
    % Document parameters
    \title{lesson1}
    
    
    

    % Pygments definitions
    
\makeatletter
\def\PY@reset{\let\PY@it=\relax \let\PY@bf=\relax%
    \let\PY@ul=\relax \let\PY@tc=\relax%
    \let\PY@bc=\relax \let\PY@ff=\relax}
\def\PY@tok#1{\csname PY@tok@#1\endcsname}
\def\PY@toks#1+{\ifx\relax#1\empty\else%
    \PY@tok{#1}\expandafter\PY@toks\fi}
\def\PY@do#1{\PY@bc{\PY@tc{\PY@ul{%
    \PY@it{\PY@bf{\PY@ff{#1}}}}}}}
\def\PY#1#2{\PY@reset\PY@toks#1+\relax+\PY@do{#2}}

\expandafter\def\csname PY@tok@w\endcsname{\def\PY@tc##1{\textcolor[rgb]{0.73,0.73,0.73}{##1}}}
\expandafter\def\csname PY@tok@c\endcsname{\let\PY@it=\textit\def\PY@tc##1{\textcolor[rgb]{0.25,0.50,0.50}{##1}}}
\expandafter\def\csname PY@tok@cp\endcsname{\def\PY@tc##1{\textcolor[rgb]{0.74,0.48,0.00}{##1}}}
\expandafter\def\csname PY@tok@k\endcsname{\let\PY@bf=\textbf\def\PY@tc##1{\textcolor[rgb]{0.00,0.50,0.00}{##1}}}
\expandafter\def\csname PY@tok@kp\endcsname{\def\PY@tc##1{\textcolor[rgb]{0.00,0.50,0.00}{##1}}}
\expandafter\def\csname PY@tok@kt\endcsname{\def\PY@tc##1{\textcolor[rgb]{0.69,0.00,0.25}{##1}}}
\expandafter\def\csname PY@tok@o\endcsname{\def\PY@tc##1{\textcolor[rgb]{0.40,0.40,0.40}{##1}}}
\expandafter\def\csname PY@tok@ow\endcsname{\let\PY@bf=\textbf\def\PY@tc##1{\textcolor[rgb]{0.67,0.13,1.00}{##1}}}
\expandafter\def\csname PY@tok@nb\endcsname{\def\PY@tc##1{\textcolor[rgb]{0.00,0.50,0.00}{##1}}}
\expandafter\def\csname PY@tok@nf\endcsname{\def\PY@tc##1{\textcolor[rgb]{0.00,0.00,1.00}{##1}}}
\expandafter\def\csname PY@tok@nc\endcsname{\let\PY@bf=\textbf\def\PY@tc##1{\textcolor[rgb]{0.00,0.00,1.00}{##1}}}
\expandafter\def\csname PY@tok@nn\endcsname{\let\PY@bf=\textbf\def\PY@tc##1{\textcolor[rgb]{0.00,0.00,1.00}{##1}}}
\expandafter\def\csname PY@tok@ne\endcsname{\let\PY@bf=\textbf\def\PY@tc##1{\textcolor[rgb]{0.82,0.25,0.23}{##1}}}
\expandafter\def\csname PY@tok@nv\endcsname{\def\PY@tc##1{\textcolor[rgb]{0.10,0.09,0.49}{##1}}}
\expandafter\def\csname PY@tok@no\endcsname{\def\PY@tc##1{\textcolor[rgb]{0.53,0.00,0.00}{##1}}}
\expandafter\def\csname PY@tok@nl\endcsname{\def\PY@tc##1{\textcolor[rgb]{0.63,0.63,0.00}{##1}}}
\expandafter\def\csname PY@tok@ni\endcsname{\let\PY@bf=\textbf\def\PY@tc##1{\textcolor[rgb]{0.60,0.60,0.60}{##1}}}
\expandafter\def\csname PY@tok@na\endcsname{\def\PY@tc##1{\textcolor[rgb]{0.49,0.56,0.16}{##1}}}
\expandafter\def\csname PY@tok@nt\endcsname{\let\PY@bf=\textbf\def\PY@tc##1{\textcolor[rgb]{0.00,0.50,0.00}{##1}}}
\expandafter\def\csname PY@tok@nd\endcsname{\def\PY@tc##1{\textcolor[rgb]{0.67,0.13,1.00}{##1}}}
\expandafter\def\csname PY@tok@s\endcsname{\def\PY@tc##1{\textcolor[rgb]{0.73,0.13,0.13}{##1}}}
\expandafter\def\csname PY@tok@sd\endcsname{\let\PY@it=\textit\def\PY@tc##1{\textcolor[rgb]{0.73,0.13,0.13}{##1}}}
\expandafter\def\csname PY@tok@si\endcsname{\let\PY@bf=\textbf\def\PY@tc##1{\textcolor[rgb]{0.73,0.40,0.53}{##1}}}
\expandafter\def\csname PY@tok@se\endcsname{\let\PY@bf=\textbf\def\PY@tc##1{\textcolor[rgb]{0.73,0.40,0.13}{##1}}}
\expandafter\def\csname PY@tok@sr\endcsname{\def\PY@tc##1{\textcolor[rgb]{0.73,0.40,0.53}{##1}}}
\expandafter\def\csname PY@tok@ss\endcsname{\def\PY@tc##1{\textcolor[rgb]{0.10,0.09,0.49}{##1}}}
\expandafter\def\csname PY@tok@sx\endcsname{\def\PY@tc##1{\textcolor[rgb]{0.00,0.50,0.00}{##1}}}
\expandafter\def\csname PY@tok@m\endcsname{\def\PY@tc##1{\textcolor[rgb]{0.40,0.40,0.40}{##1}}}
\expandafter\def\csname PY@tok@gh\endcsname{\let\PY@bf=\textbf\def\PY@tc##1{\textcolor[rgb]{0.00,0.00,0.50}{##1}}}
\expandafter\def\csname PY@tok@gu\endcsname{\let\PY@bf=\textbf\def\PY@tc##1{\textcolor[rgb]{0.50,0.00,0.50}{##1}}}
\expandafter\def\csname PY@tok@gd\endcsname{\def\PY@tc##1{\textcolor[rgb]{0.63,0.00,0.00}{##1}}}
\expandafter\def\csname PY@tok@gi\endcsname{\def\PY@tc##1{\textcolor[rgb]{0.00,0.63,0.00}{##1}}}
\expandafter\def\csname PY@tok@gr\endcsname{\def\PY@tc##1{\textcolor[rgb]{1.00,0.00,0.00}{##1}}}
\expandafter\def\csname PY@tok@ge\endcsname{\let\PY@it=\textit}
\expandafter\def\csname PY@tok@gs\endcsname{\let\PY@bf=\textbf}
\expandafter\def\csname PY@tok@gp\endcsname{\let\PY@bf=\textbf\def\PY@tc##1{\textcolor[rgb]{0.00,0.00,0.50}{##1}}}
\expandafter\def\csname PY@tok@go\endcsname{\def\PY@tc##1{\textcolor[rgb]{0.53,0.53,0.53}{##1}}}
\expandafter\def\csname PY@tok@gt\endcsname{\def\PY@tc##1{\textcolor[rgb]{0.00,0.27,0.87}{##1}}}
\expandafter\def\csname PY@tok@err\endcsname{\def\PY@bc##1{\setlength{\fboxsep}{0pt}\fcolorbox[rgb]{1.00,0.00,0.00}{1,1,1}{\strut ##1}}}
\expandafter\def\csname PY@tok@kc\endcsname{\let\PY@bf=\textbf\def\PY@tc##1{\textcolor[rgb]{0.00,0.50,0.00}{##1}}}
\expandafter\def\csname PY@tok@kd\endcsname{\let\PY@bf=\textbf\def\PY@tc##1{\textcolor[rgb]{0.00,0.50,0.00}{##1}}}
\expandafter\def\csname PY@tok@kn\endcsname{\let\PY@bf=\textbf\def\PY@tc##1{\textcolor[rgb]{0.00,0.50,0.00}{##1}}}
\expandafter\def\csname PY@tok@kr\endcsname{\let\PY@bf=\textbf\def\PY@tc##1{\textcolor[rgb]{0.00,0.50,0.00}{##1}}}
\expandafter\def\csname PY@tok@bp\endcsname{\def\PY@tc##1{\textcolor[rgb]{0.00,0.50,0.00}{##1}}}
\expandafter\def\csname PY@tok@fm\endcsname{\def\PY@tc##1{\textcolor[rgb]{0.00,0.00,1.00}{##1}}}
\expandafter\def\csname PY@tok@vc\endcsname{\def\PY@tc##1{\textcolor[rgb]{0.10,0.09,0.49}{##1}}}
\expandafter\def\csname PY@tok@vg\endcsname{\def\PY@tc##1{\textcolor[rgb]{0.10,0.09,0.49}{##1}}}
\expandafter\def\csname PY@tok@vi\endcsname{\def\PY@tc##1{\textcolor[rgb]{0.10,0.09,0.49}{##1}}}
\expandafter\def\csname PY@tok@vm\endcsname{\def\PY@tc##1{\textcolor[rgb]{0.10,0.09,0.49}{##1}}}
\expandafter\def\csname PY@tok@sa\endcsname{\def\PY@tc##1{\textcolor[rgb]{0.73,0.13,0.13}{##1}}}
\expandafter\def\csname PY@tok@sb\endcsname{\def\PY@tc##1{\textcolor[rgb]{0.73,0.13,0.13}{##1}}}
\expandafter\def\csname PY@tok@sc\endcsname{\def\PY@tc##1{\textcolor[rgb]{0.73,0.13,0.13}{##1}}}
\expandafter\def\csname PY@tok@dl\endcsname{\def\PY@tc##1{\textcolor[rgb]{0.73,0.13,0.13}{##1}}}
\expandafter\def\csname PY@tok@s2\endcsname{\def\PY@tc##1{\textcolor[rgb]{0.73,0.13,0.13}{##1}}}
\expandafter\def\csname PY@tok@sh\endcsname{\def\PY@tc##1{\textcolor[rgb]{0.73,0.13,0.13}{##1}}}
\expandafter\def\csname PY@tok@s1\endcsname{\def\PY@tc##1{\textcolor[rgb]{0.73,0.13,0.13}{##1}}}
\expandafter\def\csname PY@tok@mb\endcsname{\def\PY@tc##1{\textcolor[rgb]{0.40,0.40,0.40}{##1}}}
\expandafter\def\csname PY@tok@mf\endcsname{\def\PY@tc##1{\textcolor[rgb]{0.40,0.40,0.40}{##1}}}
\expandafter\def\csname PY@tok@mh\endcsname{\def\PY@tc##1{\textcolor[rgb]{0.40,0.40,0.40}{##1}}}
\expandafter\def\csname PY@tok@mi\endcsname{\def\PY@tc##1{\textcolor[rgb]{0.40,0.40,0.40}{##1}}}
\expandafter\def\csname PY@tok@il\endcsname{\def\PY@tc##1{\textcolor[rgb]{0.40,0.40,0.40}{##1}}}
\expandafter\def\csname PY@tok@mo\endcsname{\def\PY@tc##1{\textcolor[rgb]{0.40,0.40,0.40}{##1}}}
\expandafter\def\csname PY@tok@ch\endcsname{\let\PY@it=\textit\def\PY@tc##1{\textcolor[rgb]{0.25,0.50,0.50}{##1}}}
\expandafter\def\csname PY@tok@cm\endcsname{\let\PY@it=\textit\def\PY@tc##1{\textcolor[rgb]{0.25,0.50,0.50}{##1}}}
\expandafter\def\csname PY@tok@cpf\endcsname{\let\PY@it=\textit\def\PY@tc##1{\textcolor[rgb]{0.25,0.50,0.50}{##1}}}
\expandafter\def\csname PY@tok@c1\endcsname{\let\PY@it=\textit\def\PY@tc##1{\textcolor[rgb]{0.25,0.50,0.50}{##1}}}
\expandafter\def\csname PY@tok@cs\endcsname{\let\PY@it=\textit\def\PY@tc##1{\textcolor[rgb]{0.25,0.50,0.50}{##1}}}

\def\PYZbs{\char`\\}
\def\PYZus{\char`\_}
\def\PYZob{\char`\{}
\def\PYZcb{\char`\}}
\def\PYZca{\char`\^}
\def\PYZam{\char`\&}
\def\PYZlt{\char`\<}
\def\PYZgt{\char`\>}
\def\PYZsh{\char`\#}
\def\PYZpc{\char`\%}
\def\PYZdl{\char`\$}
\def\PYZhy{\char`\-}
\def\PYZsq{\char`\'}
\def\PYZdq{\char`\"}
\def\PYZti{\char`\~}
% for compatibility with earlier versions
\def\PYZat{@}
\def\PYZlb{[}
\def\PYZrb{]}
\makeatother


    % Exact colors from NB
    \definecolor{incolor}{rgb}{0.0, 0.0, 0.5}
    \definecolor{outcolor}{rgb}{0.545, 0.0, 0.0}



    
    % Prevent overflowing lines due to hard-to-break entities
    \sloppy 
    % Setup hyperref package
    \hypersetup{
      breaklinks=true,  % so long urls are correctly broken across lines
      colorlinks=true,
      urlcolor=urlcolor,
      linkcolor=linkcolor,
      citecolor=citecolor,
      }
    % Slightly bigger margins than the latex defaults
    
    \geometry{verbose,tmargin=1in,bmargin=1in,lmargin=1in,rmargin=1in}
    
    

    \begin{document}
    
    
    \maketitle
    
    

    
    \subsection{Image classification with Convolutional Neural
Networks}\label{image-classification-with-convolutional-neural-networks}

    Welcome to the first week of the second deep learning certificate! We're
going to use convolutional neural networks (CNNs) to allow our computer
to see - something that is only possible thanks to deep learning.

    \subsection{Introduction to our first task: 'Dogs vs
Cats'}\label{introduction-to-our-first-task-dogs-vs-cats}

    We're going to try to create a model to enter the Dogs vs Cats
competition at Kaggle. There are 25,000 labelled dog and cat photos
available for training, and 12,500 in the test set that we have to try
to label for this competition. According to the Kaggle web-site, when
this competition was launched (end of 2013): "State of the art: The
current literature suggests machine classifiers can score above 80\%
accuracy on this task". So if we can beat 80\%, then we will be at the
cutting edge as of 2013!

    \begin{Verbatim}[commandchars=\\\{\}]
{\color{incolor}In [{\color{incolor}3}]:} \PY{c+c1}{\PYZsh{} Put these at the top of every notebook, to get automatic reloading and inline plotting}
        \PY{o}{\PYZpc{}}\PY{k}{reload\PYZus{}ext} autoreload
        \PY{o}{\PYZpc{}}\PY{k}{autoreload} 2
        \PY{o}{\PYZpc{}}\PY{k}{matplotlib} inline
\end{Verbatim}


    Here we import the libraries we need. We'll learn about what each does
during the course.

    \begin{Verbatim}[commandchars=\\\{\}]
{\color{incolor}In [{\color{incolor}4}]:} \PY{c+c1}{\PYZsh{} This file contains all the main external libs we\PYZsq{}ll use}
        \PY{k+kn}{from} \PY{n+nn}{fastai}\PY{n+nn}{.}\PY{n+nn}{imports} \PY{k}{import} \PY{o}{*}
\end{Verbatim}


    \begin{Verbatim}[commandchars=\\\{\}]
{\color{incolor}In [{\color{incolor}5}]:} \PY{k+kn}{from} \PY{n+nn}{fastai}\PY{n+nn}{.}\PY{n+nn}{transforms} \PY{k}{import} \PY{o}{*}
        \PY{k+kn}{from} \PY{n+nn}{fastai}\PY{n+nn}{.}\PY{n+nn}{conv\PYZus{}learner} \PY{k}{import} \PY{o}{*}
        \PY{k+kn}{from} \PY{n+nn}{fastai}\PY{n+nn}{.}\PY{n+nn}{model} \PY{k}{import} \PY{o}{*}
        \PY{k+kn}{from} \PY{n+nn}{fastai}\PY{n+nn}{.}\PY{n+nn}{dataset} \PY{k}{import} \PY{o}{*}
        \PY{k+kn}{from} \PY{n+nn}{fastai}\PY{n+nn}{.}\PY{n+nn}{sgdr} \PY{k}{import} \PY{o}{*}
        \PY{k+kn}{from} \PY{n+nn}{fastai}\PY{n+nn}{.}\PY{n+nn}{plots} \PY{k}{import} \PY{o}{*}
\end{Verbatim}


    \texttt{PATH} is the path to your data - if you use the recommended
setup approaches from the lesson, you won't need to change this.
\texttt{sz} is the size that the images will be resized to in order to
ensure that the training runs quickly. We'll be talking about this
parameter a lot during the course. Leave it at \texttt{224} for now.

    \begin{Verbatim}[commandchars=\\\{\}]
{\color{incolor}In [{\color{incolor}6}]:} \PY{n}{PATH} \PY{o}{=} \PY{l+s+s2}{\PYZdq{}}\PY{l+s+s2}{data/dogscats/}\PY{l+s+s2}{\PYZdq{}}
        \PY{n}{sz}\PY{o}{=}\PY{l+m+mi}{224}
\end{Verbatim}


    It's important that you have a working NVidia GPU set up. The
programming framework used to behind the scenes to work with NVidia GPUs
is called CUDA. Therefore, you need to ensure the following line returns
\texttt{True} before you proceed. If you have problems with this, please
check the FAQ and ask for help on \href{http://forums.fast.ai}{the
forums}.

    \begin{Verbatim}[commandchars=\\\{\}]
{\color{incolor}In [{\color{incolor}7}]:} \PY{n}{torch}\PY{o}{.}\PY{n}{cuda}\PY{o}{.}\PY{n}{is\PYZus{}available}\PY{p}{(}\PY{p}{)}
\end{Verbatim}


\begin{Verbatim}[commandchars=\\\{\}]
{\color{outcolor}Out[{\color{outcolor}7}]:} True
\end{Verbatim}
            
    In addition, NVidia provides special accelerated functions for deep
learning in a package called CuDNN. Although not strictly necessary, it
will improve training performance significantly, and is included by
default in all supported fastai configurations. Therefore, if the
following does not return \texttt{True}, you may want to look into why.

    \begin{Verbatim}[commandchars=\\\{\}]
{\color{incolor}In [{\color{incolor}8}]:} \PY{n}{torch}\PY{o}{.}\PY{n}{backends}\PY{o}{.}\PY{n}{cudnn}\PY{o}{.}\PY{n}{enabled}
\end{Verbatim}


\begin{Verbatim}[commandchars=\\\{\}]
{\color{outcolor}Out[{\color{outcolor}8}]:} True
\end{Verbatim}
            
    \subsubsection{Extra steps if NOT using Crestle or Paperspace or our
scripts}\label{extra-steps-if-not-using-crestle-or-paperspace-or-our-scripts}

    The dataset is available at http://files.fast.ai/data/dogscats.zip. You
can download it directly on your server by running the following line in
your terminal. \texttt{wget\ http://files.fast.ai/data/dogscats.zip}.
You should put the data in a subdirectory of this notebook's directory,
called \texttt{data/}. Note that this data is already available in
Crestle and the Paperspace fast.ai template.

    \subsubsection{Extra steps if using
Crestle}\label{extra-steps-if-using-crestle}

    Crestle has the datasets required for fast.ai in /datasets, so we'll
create symlinks to the data we want for this competition. (NB: we can't
write to /datasets, but we need a place to store temporary files, so we
create our own writable directory to put the symlinks in, and we also
take advantage of Crestle's \texttt{/cache/} faster temporary storage
space.)

To run these commands (\textbf{which you should only do if using
Crestle}) remove the \texttt{\#} characters from the start of each line.

    \begin{Verbatim}[commandchars=\\\{\}]
{\color{incolor}In [{\color{incolor}6}]:} \PY{c+c1}{\PYZsh{} os.makedirs(\PYZsq{}data/dogscats/models\PYZsq{}, exist\PYZus{}ok=True)}
        
        \PY{c+c1}{\PYZsh{} !ln \PYZhy{}s /datasets/fast.ai/dogscats/train \PYZob{}PATH\PYZcb{}}
        \PY{c+c1}{\PYZsh{} !ln \PYZhy{}s /datasets/fast.ai/dogscats/test \PYZob{}PATH\PYZcb{}}
        \PY{c+c1}{\PYZsh{} !ln \PYZhy{}s /datasets/fast.ai/dogscats/valid \PYZob{}PATH\PYZcb{}}
        
        \PY{c+c1}{\PYZsh{} os.makedirs(\PYZsq{}/cache/tmp\PYZsq{}, exist\PYZus{}ok=True)}
        \PY{c+c1}{\PYZsh{} !ln \PYZhy{}fs /cache/tmp \PYZob{}PATH\PYZcb{}}
\end{Verbatim}


    \begin{Verbatim}[commandchars=\\\{\}]
{\color{incolor}In [{\color{incolor}6}]:} \PY{c+c1}{\PYZsh{} os.makedirs(\PYZsq{}/cache/tmp\PYZsq{}, exist\PYZus{}ok=True)}
        \PY{c+c1}{\PYZsh{} !ln \PYZhy{}fs /cache/tmp \PYZob{}PATH\PYZcb{}}
\end{Verbatim}


    \subsection{First look at cat
pictures}\label{first-look-at-cat-pictures}

    Our library will assume that you have \emph{train} and \emph{valid}
directories. It also assumes that each dir will have subdirs for each
class you wish to recognize (in this case, 'cats' and 'dogs').

    \begin{Verbatim}[commandchars=\\\{\}]
{\color{incolor}In [{\color{incolor}6}]:} \PY{n}{os}\PY{o}{.}\PY{n}{listdir}\PY{p}{(}\PY{n}{PATH}\PY{p}{)}
\end{Verbatim}


\begin{Verbatim}[commandchars=\\\{\}]
{\color{outcolor}Out[{\color{outcolor}6}]:} ['models', 'sample', 'test1', 'train', 'valid']
\end{Verbatim}
            
    \begin{Verbatim}[commandchars=\\\{\}]
{\color{incolor}In [{\color{incolor}7}]:} \PY{n}{os}\PY{o}{.}\PY{n}{listdir}\PY{p}{(}\PY{n}{f}\PY{l+s+s1}{\PYZsq{}}\PY{l+s+si}{\PYZob{}PATH\PYZcb{}}\PY{l+s+s1}{valid}\PY{l+s+s1}{\PYZsq{}}\PY{p}{)}
\end{Verbatim}


\begin{Verbatim}[commandchars=\\\{\}]
{\color{outcolor}Out[{\color{outcolor}7}]:} ['cats', 'dogs']
\end{Verbatim}
            
    \begin{Verbatim}[commandchars=\\\{\}]
{\color{incolor}In [{\color{incolor}9}]:} \PY{n}{files} \PY{o}{=} \PY{n}{os}\PY{o}{.}\PY{n}{listdir}\PY{p}{(}\PY{n}{f}\PY{l+s+s1}{\PYZsq{}}\PY{l+s+si}{\PYZob{}PATH\PYZcb{}}\PY{l+s+s1}{valid/cats}\PY{l+s+s1}{\PYZsq{}}\PY{p}{)}\PY{p}{[}\PY{p}{:}\PY{l+m+mi}{5}\PY{p}{]}
        \PY{n}{files}
\end{Verbatim}


\begin{Verbatim}[commandchars=\\\{\}]
{\color{outcolor}Out[{\color{outcolor}9}]:} ['cat.1001.jpg',
         'cat.10016.jpg',
         'cat.10026.jpg',
         'cat.10048.jpg',
         'cat.10050.jpg']
\end{Verbatim}
            
    \begin{Verbatim}[commandchars=\\\{\}]
{\color{incolor}In [{\color{incolor}10}]:} \PY{n}{img} \PY{o}{=} \PY{n}{plt}\PY{o}{.}\PY{n}{imread}\PY{p}{(}\PY{n}{f}\PY{l+s+s1}{\PYZsq{}}\PY{l+s+si}{\PYZob{}PATH\PYZcb{}}\PY{l+s+s1}{valid/cats/}\PY{l+s+si}{\PYZob{}files[0]\PYZcb{}}\PY{l+s+s1}{\PYZsq{}}\PY{p}{)}
         \PY{n}{plt}\PY{o}{.}\PY{n}{imshow}\PY{p}{(}\PY{n}{img}\PY{p}{)}\PY{p}{;}
\end{Verbatim}


    \begin{center}
    \adjustimage{max size={0.9\linewidth}{0.9\paperheight}}{output_25_0.png}
    \end{center}
    { \hspace*{\fill} \\}
    
    Here is how the raw data looks like

    \begin{Verbatim}[commandchars=\\\{\}]
{\color{incolor}In [{\color{incolor}11}]:} \PY{n}{img}\PY{o}{.}\PY{n}{shape}
\end{Verbatim}


\begin{Verbatim}[commandchars=\\\{\}]
{\color{outcolor}Out[{\color{outcolor}11}]:} (499, 336, 3)
\end{Verbatim}
            
    \begin{Verbatim}[commandchars=\\\{\}]
{\color{incolor}In [{\color{incolor}12}]:} \PY{n}{img}\PY{p}{[}\PY{p}{:}\PY{l+m+mi}{4}\PY{p}{,}\PY{p}{:}\PY{l+m+mi}{4}\PY{p}{]}
\end{Verbatim}


\begin{Verbatim}[commandchars=\\\{\}]
{\color{outcolor}Out[{\color{outcolor}12}]:} array([[[60, 58, 10],
                 [60, 57, 14],
                 [61, 56, 18],
                 [63, 54, 23]],
         
                [[56, 54,  6],
                 [56, 53, 10],
                 [57, 52, 14],
                 [60, 51, 20]],
         
                [[52, 49,  4],
                 [52, 49,  6],
                 [53, 48, 10],
                 [56, 47, 16]],
         
                [[50, 47,  2],
                 [50, 47,  4],
                 [51, 45,  9],
                 [53, 44, 13]]], dtype=uint8)
\end{Verbatim}
            
    \subsection{Our first model: quick
start}\label{our-first-model-quick-start}

    We're going to use a pre-trained model, that is, a model created by some
one else to solve a different problem. Instead of building a model from
scratch to solve a similar problem, we'll use a model trained on
ImageNet (1.2 million images and 1000 classes) as a starting point. The
model is a Convolutional Neural Network (CNN), a type of Neural Network
that builds state-of-the-art models for computer vision. We'll be
learning all about CNNs during this course.

We will be using the resnet34 model. resnet34 is a version of the model
that won the 2015 ImageNet competition. Here is more info on
\href{https://github.com/KaimingHe/deep-residual-networks}{resnet
models}. We'll be studying them in depth later, but for now we'll focus
on using them effectively.

Here's how to train and evalulate a \emph{dogs vs cats} model in 3 lines
of code, and under 20 seconds:

    \begin{Verbatim}[commandchars=\\\{\}]
{\color{incolor}In [{\color{incolor}6}]:} \PY{c+c1}{\PYZsh{} Uncomment the below if you need to reset your precomputed activations}
        \PY{c+c1}{\PYZsh{} shutil.rmtree(f\PYZsq{}\PYZob{}PATH\PYZcb{}tmp\PYZsq{}, ignore\PYZus{}errors=True)}
\end{Verbatim}


    \begin{Verbatim}[commandchars=\\\{\}]
{\color{incolor}In [{\color{incolor}9}]:} \PY{n}{arch}\PY{o}{=}\PY{n}{resnet34}
        \PY{n}{data} \PY{o}{=} \PY{n}{ImageClassifierData}\PY{o}{.}\PY{n}{from\PYZus{}paths}\PY{p}{(}\PY{n}{PATH}\PY{p}{,} \PY{n}{tfms}\PY{o}{=}\PY{n}{tfms\PYZus{}from\PYZus{}model}\PY{p}{(}\PY{n}{arch}\PY{p}{,} \PY{n}{sz}\PY{p}{)}\PY{p}{)}
        \PY{n}{learn} \PY{o}{=} \PY{n}{ConvLearner}\PY{o}{.}\PY{n}{pretrained}\PY{p}{(}\PY{n}{arch}\PY{p}{,} \PY{n}{data}\PY{p}{,} \PY{n}{precompute}\PY{o}{=}\PY{k+kc}{True}\PY{p}{)}
        \PY{n}{learn}\PY{o}{.}\PY{n}{fit}\PY{p}{(}\PY{l+m+mf}{0.01}\PY{p}{,} \PY{l+m+mi}{2}\PY{p}{)}
\end{Verbatim}


    
    \begin{verbatim}
HBox(children=(IntProgress(value=0, description='Epoch', max=2), HTML(value='')))
    \end{verbatim}

    
    \begin{Verbatim}[commandchars=\\\{\}]
epoch      trn\_loss   val\_loss   accuracy                                                                              
    0      0.042222   0.028351   0.991211  
    1      0.035367   0.026421   0.991211                                                                              


    \end{Verbatim}

\begin{Verbatim}[commandchars=\\\{\}]
{\color{outcolor}Out[{\color{outcolor}9}]:} [0.026420766, 0.9912109375]
\end{Verbatim}
            
    How good is this model? Well, as we mentioned, prior to this
competition, the state of the art was 80\% accuracy. But the competition
resulted in a huge jump to 98.9\% accuracy, with the author of a popular
deep learning library winning the competition. Extraordinarily, less
than 4 years later, we can now beat that result in seconds! Even last
year in this same course, our initial model had 98.3\% accuracy, which
is nearly double the error we're getting just a year later, and that
took around 10 minutes to compute.

    \subsection{Analyzing results: looking at
pictures}\label{analyzing-results-looking-at-pictures}

    As well as looking at the overall metrics, it's also a good idea to look
at examples of each of: 1. A few correct labels at random 2. A few
incorrect labels at random 3. The most correct labels of each class (ie
those with highest probability that are correct) 4. The most incorrect
labels of each class (ie those with highest probability that are
incorrect) 5. The most uncertain labels (ie those with probability
closest to 0.5).

    \begin{Verbatim}[commandchars=\\\{\}]
{\color{incolor}In [{\color{incolor}14}]:} \PY{c+c1}{\PYZsh{} This is the label for a val data}
         \PY{n}{data}\PY{o}{.}\PY{n}{val\PYZus{}y}
\end{Verbatim}


\begin{Verbatim}[commandchars=\\\{\}]
{\color{outcolor}Out[{\color{outcolor}14}]:} array([0, 0, 0, {\ldots}, 1, 1, 1])
\end{Verbatim}
            
    \begin{Verbatim}[commandchars=\\\{\}]
{\color{incolor}In [{\color{incolor}15}]:} \PY{c+c1}{\PYZsh{} from here we know that \PYZsq{}cats\PYZsq{} is label 0 and \PYZsq{}dogs\PYZsq{} is label 1.}
         \PY{n}{data}\PY{o}{.}\PY{n}{classes}
\end{Verbatim}


\begin{Verbatim}[commandchars=\\\{\}]
{\color{outcolor}Out[{\color{outcolor}15}]:} ['cats', 'dogs']
\end{Verbatim}
            
    \begin{Verbatim}[commandchars=\\\{\}]
{\color{incolor}In [{\color{incolor}16}]:} \PY{c+c1}{\PYZsh{} this gives prediction for validation set. Predictions are in log scale}
         \PY{n}{log\PYZus{}preds} \PY{o}{=} \PY{n}{learn}\PY{o}{.}\PY{n}{predict}\PY{p}{(}\PY{p}{)}
         \PY{n}{log\PYZus{}preds}\PY{o}{.}\PY{n}{shape}
\end{Verbatim}


\begin{Verbatim}[commandchars=\\\{\}]
{\color{outcolor}Out[{\color{outcolor}16}]:} (2000, 2)
\end{Verbatim}
            
    \begin{Verbatim}[commandchars=\\\{\}]
{\color{incolor}In [{\color{incolor}17}]:} \PY{n}{log\PYZus{}preds}\PY{p}{[}\PY{p}{:}\PY{l+m+mi}{10}\PY{p}{]}
\end{Verbatim}


\begin{Verbatim}[commandchars=\\\{\}]
{\color{outcolor}Out[{\color{outcolor}17}]:} array([[ -0.00002, -11.07446],
                [ -0.00138,  -6.58385],
                [ -0.00083,  -7.09025],
                [ -0.00029,  -8.13645],
                [ -0.00035,  -7.9663 ],
                [ -0.00029,  -8.15125],
                [ -0.00002, -10.82139],
                [ -0.00003, -10.33846],
                [ -0.00323,  -5.73731],
                [ -0.0001 ,  -9.21326]], dtype=float32)
\end{Verbatim}
            
    \begin{Verbatim}[commandchars=\\\{\}]
{\color{incolor}In [{\color{incolor}18}]:} \PY{n}{preds} \PY{o}{=} \PY{n}{np}\PY{o}{.}\PY{n}{argmax}\PY{p}{(}\PY{n}{log\PYZus{}preds}\PY{p}{,} \PY{n}{axis}\PY{o}{=}\PY{l+m+mi}{1}\PY{p}{)}  \PY{c+c1}{\PYZsh{} from log probabilities to 0 or 1}
         \PY{n}{probs} \PY{o}{=} \PY{n}{np}\PY{o}{.}\PY{n}{exp}\PY{p}{(}\PY{n}{log\PYZus{}preds}\PY{p}{[}\PY{p}{:}\PY{p}{,}\PY{l+m+mi}{1}\PY{p}{]}\PY{p}{)}        \PY{c+c1}{\PYZsh{} pr(dog)}
\end{Verbatim}


    \begin{Verbatim}[commandchars=\\\{\}]
{\color{incolor}In [{\color{incolor}19}]:} \PY{k}{def} \PY{n+nf}{rand\PYZus{}by\PYZus{}mask}\PY{p}{(}\PY{n}{mask}\PY{p}{)}\PY{p}{:} \PY{k}{return} \PY{n}{np}\PY{o}{.}\PY{n}{random}\PY{o}{.}\PY{n}{choice}\PY{p}{(}\PY{n}{np}\PY{o}{.}\PY{n}{where}\PY{p}{(}\PY{n}{mask}\PY{p}{)}\PY{p}{[}\PY{l+m+mi}{0}\PY{p}{]}\PY{p}{,} \PY{l+m+mi}{4}\PY{p}{,} \PY{n}{replace}\PY{o}{=}\PY{k+kc}{False}\PY{p}{)}
         \PY{k}{def} \PY{n+nf}{rand\PYZus{}by\PYZus{}correct}\PY{p}{(}\PY{n}{is\PYZus{}correct}\PY{p}{)}\PY{p}{:} \PY{k}{return} \PY{n}{rand\PYZus{}by\PYZus{}mask}\PY{p}{(}\PY{p}{(}\PY{n}{preds} \PY{o}{==} \PY{n}{data}\PY{o}{.}\PY{n}{val\PYZus{}y}\PY{p}{)}\PY{o}{==}\PY{n}{is\PYZus{}correct}\PY{p}{)}
\end{Verbatim}


    \begin{Verbatim}[commandchars=\\\{\}]
{\color{incolor}In [{\color{incolor}20}]:} \PY{k}{def} \PY{n+nf}{plot\PYZus{}val\PYZus{}with\PYZus{}title}\PY{p}{(}\PY{n}{idxs}\PY{p}{,} \PY{n}{title}\PY{p}{)}\PY{p}{:}
             \PY{n}{imgs} \PY{o}{=} \PY{n}{np}\PY{o}{.}\PY{n}{stack}\PY{p}{(}\PY{p}{[}\PY{n}{data}\PY{o}{.}\PY{n}{val\PYZus{}ds}\PY{p}{[}\PY{n}{x}\PY{p}{]}\PY{p}{[}\PY{l+m+mi}{0}\PY{p}{]} \PY{k}{for} \PY{n}{x} \PY{o+ow}{in} \PY{n}{idxs}\PY{p}{]}\PY{p}{)}
             \PY{n}{title\PYZus{}probs} \PY{o}{=} \PY{p}{[}\PY{n}{probs}\PY{p}{[}\PY{n}{x}\PY{p}{]} \PY{k}{for} \PY{n}{x} \PY{o+ow}{in} \PY{n}{idxs}\PY{p}{]}
             \PY{n+nb}{print}\PY{p}{(}\PY{n}{title}\PY{p}{)}
             \PY{k}{return} \PY{n}{plots}\PY{p}{(}\PY{n}{data}\PY{o}{.}\PY{n}{val\PYZus{}ds}\PY{o}{.}\PY{n}{denorm}\PY{p}{(}\PY{n}{imgs}\PY{p}{)}\PY{p}{,} \PY{n}{rows}\PY{o}{=}\PY{l+m+mi}{1}\PY{p}{,} \PY{n}{titles}\PY{o}{=}\PY{n}{title\PYZus{}probs}\PY{p}{)}
\end{Verbatim}


    \begin{Verbatim}[commandchars=\\\{\}]
{\color{incolor}In [{\color{incolor}21}]:} \PY{k}{def} \PY{n+nf}{plots}\PY{p}{(}\PY{n}{ims}\PY{p}{,} \PY{n}{figsize}\PY{o}{=}\PY{p}{(}\PY{l+m+mi}{12}\PY{p}{,}\PY{l+m+mi}{6}\PY{p}{)}\PY{p}{,} \PY{n}{rows}\PY{o}{=}\PY{l+m+mi}{1}\PY{p}{,} \PY{n}{titles}\PY{o}{=}\PY{k+kc}{None}\PY{p}{)}\PY{p}{:}
             \PY{n}{f} \PY{o}{=} \PY{n}{plt}\PY{o}{.}\PY{n}{figure}\PY{p}{(}\PY{n}{figsize}\PY{o}{=}\PY{n}{figsize}\PY{p}{)}
             \PY{k}{for} \PY{n}{i} \PY{o+ow}{in} \PY{n+nb}{range}\PY{p}{(}\PY{n+nb}{len}\PY{p}{(}\PY{n}{ims}\PY{p}{)}\PY{p}{)}\PY{p}{:}
                 \PY{n}{sp} \PY{o}{=} \PY{n}{f}\PY{o}{.}\PY{n}{add\PYZus{}subplot}\PY{p}{(}\PY{n}{rows}\PY{p}{,} \PY{n+nb}{len}\PY{p}{(}\PY{n}{ims}\PY{p}{)}\PY{o}{/}\PY{o}{/}\PY{n}{rows}\PY{p}{,} \PY{n}{i}\PY{o}{+}\PY{l+m+mi}{1}\PY{p}{)}
                 \PY{n}{sp}\PY{o}{.}\PY{n}{axis}\PY{p}{(}\PY{l+s+s1}{\PYZsq{}}\PY{l+s+s1}{Off}\PY{l+s+s1}{\PYZsq{}}\PY{p}{)}
                 \PY{k}{if} \PY{n}{titles} \PY{o+ow}{is} \PY{o+ow}{not} \PY{k+kc}{None}\PY{p}{:} \PY{n}{sp}\PY{o}{.}\PY{n}{set\PYZus{}title}\PY{p}{(}\PY{n}{titles}\PY{p}{[}\PY{n}{i}\PY{p}{]}\PY{p}{,} \PY{n}{fontsize}\PY{o}{=}\PY{l+m+mi}{16}\PY{p}{)}
                 \PY{n}{plt}\PY{o}{.}\PY{n}{imshow}\PY{p}{(}\PY{n}{ims}\PY{p}{[}\PY{n}{i}\PY{p}{]}\PY{p}{)}
\end{Verbatim}


    \begin{Verbatim}[commandchars=\\\{\}]
{\color{incolor}In [{\color{incolor}22}]:} \PY{k}{def} \PY{n+nf}{load\PYZus{}img\PYZus{}id}\PY{p}{(}\PY{n}{ds}\PY{p}{,} \PY{n}{idx}\PY{p}{)}\PY{p}{:} \PY{k}{return} \PY{n}{np}\PY{o}{.}\PY{n}{array}\PY{p}{(}\PY{n}{PIL}\PY{o}{.}\PY{n}{Image}\PY{o}{.}\PY{n}{open}\PY{p}{(}\PY{n}{PATH}\PY{o}{+}\PY{n}{ds}\PY{o}{.}\PY{n}{fnames}\PY{p}{[}\PY{n}{idx}\PY{p}{]}\PY{p}{)}\PY{p}{)}
         
         \PY{k}{def} \PY{n+nf}{plot\PYZus{}val\PYZus{}with\PYZus{}title}\PY{p}{(}\PY{n}{idxs}\PY{p}{,} \PY{n}{title}\PY{p}{)}\PY{p}{:}
             \PY{n}{imgs} \PY{o}{=} \PY{p}{[}\PY{n}{load\PYZus{}img\PYZus{}id}\PY{p}{(}\PY{n}{data}\PY{o}{.}\PY{n}{val\PYZus{}ds}\PY{p}{,}\PY{n}{x}\PY{p}{)} \PY{k}{for} \PY{n}{x} \PY{o+ow}{in} \PY{n}{idxs}\PY{p}{]}
             \PY{n}{title\PYZus{}probs} \PY{o}{=} \PY{p}{[}\PY{n}{probs}\PY{p}{[}\PY{n}{x}\PY{p}{]} \PY{k}{for} \PY{n}{x} \PY{o+ow}{in} \PY{n}{idxs}\PY{p}{]}
             \PY{n+nb}{print}\PY{p}{(}\PY{n}{title}\PY{p}{)}
             \PY{k}{return} \PY{n}{plots}\PY{p}{(}\PY{n}{imgs}\PY{p}{,} \PY{n}{rows}\PY{o}{=}\PY{l+m+mi}{1}\PY{p}{,} \PY{n}{titles}\PY{o}{=}\PY{n}{title\PYZus{}probs}\PY{p}{,} \PY{n}{figsize}\PY{o}{=}\PY{p}{(}\PY{l+m+mi}{16}\PY{p}{,}\PY{l+m+mi}{8}\PY{p}{)}\PY{p}{)}
\end{Verbatim}


    \begin{Verbatim}[commandchars=\\\{\}]
{\color{incolor}In [{\color{incolor}23}]:} \PY{c+c1}{\PYZsh{} 1. A few correct labels at random}
         \PY{n}{plot\PYZus{}val\PYZus{}with\PYZus{}title}\PY{p}{(}\PY{n}{rand\PYZus{}by\PYZus{}correct}\PY{p}{(}\PY{k+kc}{True}\PY{p}{)}\PY{p}{,} \PY{l+s+s2}{\PYZdq{}}\PY{l+s+s2}{Correctly classified}\PY{l+s+s2}{\PYZdq{}}\PY{p}{)}
\end{Verbatim}


    \begin{Verbatim}[commandchars=\\\{\}]
Correctly classified

    \end{Verbatim}

    \begin{center}
    \adjustimage{max size={0.9\linewidth}{0.9\paperheight}}{output_45_1.png}
    \end{center}
    { \hspace*{\fill} \\}
    
    \begin{Verbatim}[commandchars=\\\{\}]
{\color{incolor}In [{\color{incolor}24}]:} \PY{c+c1}{\PYZsh{} 2. A few incorrect labels at random}
         \PY{n}{plot\PYZus{}val\PYZus{}with\PYZus{}title}\PY{p}{(}\PY{n}{rand\PYZus{}by\PYZus{}correct}\PY{p}{(}\PY{k+kc}{False}\PY{p}{)}\PY{p}{,} \PY{l+s+s2}{\PYZdq{}}\PY{l+s+s2}{Incorrectly classified}\PY{l+s+s2}{\PYZdq{}}\PY{p}{)}
\end{Verbatim}


    \begin{Verbatim}[commandchars=\\\{\}]
Incorrectly classified

    \end{Verbatim}

    \begin{center}
    \adjustimage{max size={0.9\linewidth}{0.9\paperheight}}{output_46_1.png}
    \end{center}
    { \hspace*{\fill} \\}
    
    \begin{Verbatim}[commandchars=\\\{\}]
{\color{incolor}In [{\color{incolor}25}]:} \PY{k}{def} \PY{n+nf}{most\PYZus{}by\PYZus{}mask}\PY{p}{(}\PY{n}{mask}\PY{p}{,} \PY{n}{mult}\PY{p}{)}\PY{p}{:}
             \PY{n}{idxs} \PY{o}{=} \PY{n}{np}\PY{o}{.}\PY{n}{where}\PY{p}{(}\PY{n}{mask}\PY{p}{)}\PY{p}{[}\PY{l+m+mi}{0}\PY{p}{]}
             \PY{k}{return} \PY{n}{idxs}\PY{p}{[}\PY{n}{np}\PY{o}{.}\PY{n}{argsort}\PY{p}{(}\PY{n}{mult} \PY{o}{*} \PY{n}{probs}\PY{p}{[}\PY{n}{idxs}\PY{p}{]}\PY{p}{)}\PY{p}{[}\PY{p}{:}\PY{l+m+mi}{4}\PY{p}{]}\PY{p}{]}
         
         \PY{k}{def} \PY{n+nf}{most\PYZus{}by\PYZus{}correct}\PY{p}{(}\PY{n}{y}\PY{p}{,} \PY{n}{is\PYZus{}correct}\PY{p}{)}\PY{p}{:} 
             \PY{n}{mult} \PY{o}{=} \PY{o}{\PYZhy{}}\PY{l+m+mi}{1} \PY{k}{if} \PY{p}{(}\PY{n}{y}\PY{o}{==}\PY{l+m+mi}{1}\PY{p}{)}\PY{o}{==}\PY{n}{is\PYZus{}correct} \PY{k}{else} \PY{l+m+mi}{1}
             \PY{k}{return} \PY{n}{most\PYZus{}by\PYZus{}mask}\PY{p}{(}\PY{p}{(}\PY{p}{(}\PY{n}{preds} \PY{o}{==} \PY{n}{data}\PY{o}{.}\PY{n}{val\PYZus{}y}\PY{p}{)}\PY{o}{==}\PY{n}{is\PYZus{}correct}\PY{p}{)} \PY{o}{\PYZam{}} \PY{p}{(}\PY{n}{data}\PY{o}{.}\PY{n}{val\PYZus{}y} \PY{o}{==} \PY{n}{y}\PY{p}{)}\PY{p}{,} \PY{n}{mult}\PY{p}{)}
\end{Verbatim}


    \begin{Verbatim}[commandchars=\\\{\}]
{\color{incolor}In [{\color{incolor}26}]:} \PY{n}{plot\PYZus{}val\PYZus{}with\PYZus{}title}\PY{p}{(}\PY{n}{most\PYZus{}by\PYZus{}correct}\PY{p}{(}\PY{l+m+mi}{0}\PY{p}{,} \PY{k+kc}{True}\PY{p}{)}\PY{p}{,} \PY{l+s+s2}{\PYZdq{}}\PY{l+s+s2}{Most correct cats}\PY{l+s+s2}{\PYZdq{}}\PY{p}{)}
\end{Verbatim}


    \begin{Verbatim}[commandchars=\\\{\}]
Most correct cats

    \end{Verbatim}

    \begin{center}
    \adjustimage{max size={0.9\linewidth}{0.9\paperheight}}{output_48_1.png}
    \end{center}
    { \hspace*{\fill} \\}
    
    \begin{Verbatim}[commandchars=\\\{\}]
{\color{incolor}In [{\color{incolor}27}]:} \PY{n}{plot\PYZus{}val\PYZus{}with\PYZus{}title}\PY{p}{(}\PY{n}{most\PYZus{}by\PYZus{}correct}\PY{p}{(}\PY{l+m+mi}{1}\PY{p}{,} \PY{k+kc}{True}\PY{p}{)}\PY{p}{,} \PY{l+s+s2}{\PYZdq{}}\PY{l+s+s2}{Most correct dogs}\PY{l+s+s2}{\PYZdq{}}\PY{p}{)}
\end{Verbatim}


    \begin{Verbatim}[commandchars=\\\{\}]
Most correct dogs

    \end{Verbatim}

    \begin{center}
    \adjustimage{max size={0.9\linewidth}{0.9\paperheight}}{output_49_1.png}
    \end{center}
    { \hspace*{\fill} \\}
    
    \begin{Verbatim}[commandchars=\\\{\}]
{\color{incolor}In [{\color{incolor}28}]:} \PY{n}{plot\PYZus{}val\PYZus{}with\PYZus{}title}\PY{p}{(}\PY{n}{most\PYZus{}by\PYZus{}correct}\PY{p}{(}\PY{l+m+mi}{0}\PY{p}{,} \PY{k+kc}{False}\PY{p}{)}\PY{p}{,} \PY{l+s+s2}{\PYZdq{}}\PY{l+s+s2}{Most incorrect cats}\PY{l+s+s2}{\PYZdq{}}\PY{p}{)}
\end{Verbatim}


    \begin{Verbatim}[commandchars=\\\{\}]
Most incorrect cats

    \end{Verbatim}

    \begin{center}
    \adjustimage{max size={0.9\linewidth}{0.9\paperheight}}{output_50_1.png}
    \end{center}
    { \hspace*{\fill} \\}
    
    \begin{Verbatim}[commandchars=\\\{\}]
{\color{incolor}In [{\color{incolor}29}]:} \PY{n}{plot\PYZus{}val\PYZus{}with\PYZus{}title}\PY{p}{(}\PY{n}{most\PYZus{}by\PYZus{}correct}\PY{p}{(}\PY{l+m+mi}{1}\PY{p}{,} \PY{k+kc}{False}\PY{p}{)}\PY{p}{,} \PY{l+s+s2}{\PYZdq{}}\PY{l+s+s2}{Most incorrect dogs}\PY{l+s+s2}{\PYZdq{}}\PY{p}{)}
\end{Verbatim}


    \begin{Verbatim}[commandchars=\\\{\}]
Most incorrect dogs

    \end{Verbatim}

    \begin{center}
    \adjustimage{max size={0.9\linewidth}{0.9\paperheight}}{output_51_1.png}
    \end{center}
    { \hspace*{\fill} \\}
    
    \begin{Verbatim}[commandchars=\\\{\}]
{\color{incolor}In [{\color{incolor}30}]:} \PY{n}{most\PYZus{}uncertain} \PY{o}{=} \PY{n}{np}\PY{o}{.}\PY{n}{argsort}\PY{p}{(}\PY{n}{np}\PY{o}{.}\PY{n}{abs}\PY{p}{(}\PY{n}{probs} \PY{o}{\PYZhy{}}\PY{l+m+mf}{0.5}\PY{p}{)}\PY{p}{)}\PY{p}{[}\PY{p}{:}\PY{l+m+mi}{4}\PY{p}{]}
         \PY{n}{plot\PYZus{}val\PYZus{}with\PYZus{}title}\PY{p}{(}\PY{n}{most\PYZus{}uncertain}\PY{p}{,} \PY{l+s+s2}{\PYZdq{}}\PY{l+s+s2}{Most uncertain predictions}\PY{l+s+s2}{\PYZdq{}}\PY{p}{)}
\end{Verbatim}


    \begin{Verbatim}[commandchars=\\\{\}]
Most uncertain predictions

    \end{Verbatim}

    \begin{center}
    \adjustimage{max size={0.9\linewidth}{0.9\paperheight}}{output_52_1.png}
    \end{center}
    { \hspace*{\fill} \\}
    
    \subsection{Choosing a learning rate}\label{choosing-a-learning-rate}

    The \emph{learning rate} determines how quickly or how slowly you want
to update the \emph{weights} (or \emph{parameters}). Learning rate is
one of the most difficult parameters to set, because it significantly
affect model performance.

The method \texttt{learn.lr\_find()} helps you find an optimal learning
rate. It uses the technique developed in the 2015 paper
\href{http://arxiv.org/abs/1506.01186}{Cyclical Learning Rates for
Training Neural Networks}, where we simply keep increasing the learning
rate from a very small value, until the loss starts decreasing. We can
plot the learning rate across batches to see what this looks like.

We first create a new learner, since we want to know how to set the
learning rate for a new (untrained) model.

    \begin{Verbatim}[commandchars=\\\{\}]
{\color{incolor}In [{\color{incolor}7}]:} \PY{n}{learn} \PY{o}{=} \PY{n}{ConvLearner}\PY{o}{.}\PY{n}{pretrained}\PY{p}{(}\PY{n}{arch}\PY{p}{,} \PY{n}{data}\PY{p}{,} \PY{n}{precompute}\PY{o}{=}\PY{k+kc}{True}\PY{p}{)}
\end{Verbatim}


    \begin{Verbatim}[commandchars=\\\{\}]
{\color{incolor}In [{\color{incolor}8}]:} \PY{n}{lrf}\PY{o}{=}\PY{n}{learn}\PY{o}{.}\PY{n}{lr\PYZus{}find}\PY{p}{(}\PY{p}{)}
\end{Verbatim}


    
    \begin{verbatim}
HBox(children=(IntProgress(value=0, description='Epoch', max=1), HTML(value='')))
    \end{verbatim}

    
    \begin{Verbatim}[commandchars=\\\{\}]
 84\%|████████▍ | 304/360 [00:04<00:00, 61.13it/s, loss=0.442]
                                                             
    \end{Verbatim}

    Our \texttt{learn} object contains an attribute \texttt{sched} that
contains our learning rate scheduler, and has some convenient plotting
functionality including this one:

    \begin{Verbatim}[commandchars=\\\{\}]
{\color{incolor}In [{\color{incolor}51}]:} \PY{n}{learn}\PY{o}{.}\PY{n}{sched}\PY{o}{.}\PY{n}{plot\PYZus{}lr}\PY{p}{(}\PY{p}{)}
\end{Verbatim}


    \begin{center}
    \adjustimage{max size={0.9\linewidth}{0.9\paperheight}}{output_58_0.png}
    \end{center}
    { \hspace*{\fill} \\}
    
    Note that in the previous plot \emph{iteration} is one iteration (or
\emph{minibatch}) of SGD. In one epoch there are
(num\_train\_samples/num\_iterations) of SGD.

We can see the plot of loss versus learning rate to see where our loss
stops decreasing:

    \begin{Verbatim}[commandchars=\\\{\}]
{\color{incolor}In [{\color{incolor}52}]:} \PY{n}{learn}\PY{o}{.}\PY{n}{sched}\PY{o}{.}\PY{n}{plot}\PY{p}{(}\PY{p}{)}
\end{Verbatim}


    \begin{center}
    \adjustimage{max size={0.9\linewidth}{0.9\paperheight}}{output_60_0.png}
    \end{center}
    { \hspace*{\fill} \\}
    
    The loss is still clearly improving at lr=1e-2 (0.01), so that's what we
use. Note that the optimal learning rate can change as we training the
model, so you may want to re-run this function from time to time.

    \subsection{Improving our model}\label{improving-our-model}

    \subsubsection{Data augmentation}\label{data-augmentation}

    If you try training for more epochs, you'll notice that we start to
\emph{overfit}, which means that our model is learning to recognize the
specific images in the training set, rather than generalizaing such that
we also get good results on the validation set. One way to fix this is
to effectively create more data, through \emph{data augmentation}. This
refers to randomly changing the images in ways that shouldn't impact
their interpretation, such as horizontal flipping, zooming, and
rotating.

We can do this by passing \texttt{aug\_tfms} (\emph{augmentation
transforms}) to \texttt{tfms\_from\_model}, with a list of functions to
apply that randomly change the image however we wish. For photos that
are largely taken from the side (e.g. most photos of dogs and cats, as
opposed to photos taken from the top down, such as satellite imagery) we
can use the pre-defined list of functions \texttt{transforms\_side\_on}.
We can also specify random zooming of images up to specified scale by
adding the \texttt{max\_zoom} parameter.

    \begin{Verbatim}[commandchars=\\\{\}]
{\color{incolor}In [{\color{incolor}9}]:} \PY{n}{tfms} \PY{o}{=} \PY{n}{tfms\PYZus{}from\PYZus{}model}\PY{p}{(}\PY{n}{resnet34}\PY{p}{,} \PY{n}{sz}\PY{p}{,} \PY{n}{aug\PYZus{}tfms}\PY{o}{=}\PY{n}{transforms\PYZus{}side\PYZus{}on}\PY{p}{,} \PY{n}{max\PYZus{}zoom}\PY{o}{=}\PY{l+m+mf}{1.1}\PY{p}{)}
\end{Verbatim}


    \begin{Verbatim}[commandchars=\\\{\}]
{\color{incolor}In [{\color{incolor}10}]:} \PY{k}{def} \PY{n+nf}{get\PYZus{}augs}\PY{p}{(}\PY{p}{)}\PY{p}{:}
             \PY{n}{data} \PY{o}{=} \PY{n}{ImageClassifierData}\PY{o}{.}\PY{n}{from\PYZus{}paths}\PY{p}{(}\PY{n}{PATH}\PY{p}{,} \PY{n}{bs}\PY{o}{=}\PY{l+m+mi}{2}\PY{p}{,} \PY{n}{tfms}\PY{o}{=}\PY{n}{tfms}\PY{p}{,} \PY{n}{num\PYZus{}workers}\PY{o}{=}\PY{l+m+mi}{1}\PY{p}{)}
             \PY{n}{x}\PY{p}{,}\PY{n}{\PYZus{}} \PY{o}{=} \PY{n+nb}{next}\PY{p}{(}\PY{n+nb}{iter}\PY{p}{(}\PY{n}{data}\PY{o}{.}\PY{n}{aug\PYZus{}dl}\PY{p}{)}\PY{p}{)}
             \PY{k}{return} \PY{n}{data}\PY{o}{.}\PY{n}{trn\PYZus{}ds}\PY{o}{.}\PY{n}{denorm}\PY{p}{(}\PY{n}{x}\PY{p}{)}\PY{p}{[}\PY{l+m+mi}{1}\PY{p}{]}
\end{Verbatim}


    \begin{Verbatim}[commandchars=\\\{\}]
{\color{incolor}In [{\color{incolor}27}]:} \PY{n}{ims} \PY{o}{=} \PY{n}{np}\PY{o}{.}\PY{n}{stack}\PY{p}{(}\PY{p}{[}\PY{n}{get\PYZus{}augs}\PY{p}{(}\PY{p}{)} \PY{k}{for} \PY{n}{i} \PY{o+ow}{in} \PY{n+nb}{range}\PY{p}{(}\PY{l+m+mi}{6}\PY{p}{)}\PY{p}{]}\PY{p}{)}
\end{Verbatim}


    \begin{Verbatim}[commandchars=\\\{\}]
{\color{incolor}In [{\color{incolor}28}]:} \PY{n}{plots}\PY{p}{(}\PY{n}{ims}\PY{p}{,} \PY{n}{rows}\PY{o}{=}\PY{l+m+mi}{2}\PY{p}{)}
\end{Verbatim}


    \begin{center}
    \adjustimage{max size={0.9\linewidth}{0.9\paperheight}}{output_68_0.png}
    \end{center}
    { \hspace*{\fill} \\}
    
    Let's create a new \texttt{data} object that includes this augmentation
in the transforms.

    \begin{Verbatim}[commandchars=\\\{\}]
{\color{incolor}In [{\color{incolor}11}]:} \PY{n}{data} \PY{o}{=} \PY{n}{ImageClassifierData}\PY{o}{.}\PY{n}{from\PYZus{}paths}\PY{p}{(}\PY{n}{PATH}\PY{p}{,} \PY{n}{tfms}\PY{o}{=}\PY{n}{tfms}\PY{p}{)}
         \PY{n}{learn} \PY{o}{=} \PY{n}{ConvLearner}\PY{o}{.}\PY{n}{pretrained}\PY{p}{(}\PY{n}{arch}\PY{p}{,} \PY{n}{data}\PY{p}{,} \PY{n}{precompute}\PY{o}{=}\PY{k+kc}{True}\PY{p}{)}
\end{Verbatim}


    \begin{Verbatim}[commandchars=\\\{\}]
{\color{incolor}In [{\color{incolor}12}]:} \PY{n}{learn}\PY{o}{.}\PY{n}{fit}\PY{p}{(}\PY{l+m+mf}{1e\PYZhy{}2}\PY{p}{,} \PY{l+m+mi}{1}\PY{p}{)}
\end{Verbatim}


    
    \begin{verbatim}
HBox(children=(IntProgress(value=0, description='Epoch', max=1), HTML(value='')))
    \end{verbatim}

    
    \begin{Verbatim}[commandchars=\\\{\}]
[ 0.       0.0462   0.02459  0.99121]                         


    \end{Verbatim}

    \begin{Verbatim}[commandchars=\\\{\}]
{\color{incolor}In [{\color{incolor}13}]:} \PY{n}{learn}\PY{o}{.}\PY{n}{precompute}\PY{o}{=}\PY{k+kc}{False}
\end{Verbatim}


    By default when we create a learner, it sets all but the last layer to
\emph{frozen}. That means that it's still only updating the weights in
the last layer when we call \texttt{fit}.

    \begin{Verbatim}[commandchars=\\\{\}]
{\color{incolor}In [{\color{incolor}14}]:} \PY{n}{learn}\PY{o}{.}\PY{n}{fit}\PY{p}{(}\PY{l+m+mf}{1e\PYZhy{}2}\PY{p}{,} \PY{l+m+mi}{3}\PY{p}{,} \PY{n}{cycle\PYZus{}len}\PY{o}{=}\PY{l+m+mi}{1}\PY{p}{)}
\end{Verbatim}


    
    \begin{verbatim}
HBox(children=(IntProgress(value=0, description='Epoch', max=3), HTML(value='')))
    \end{verbatim}

    
    \begin{Verbatim}[commandchars=\\\{\}]
[ 0.       0.05     0.02535  0.9917 ]                         
[ 1.       0.04248  0.02372  0.99219]                         
[ 2.       0.04918  0.02365  0.9917 ]                         


    \end{Verbatim}

    What is that \texttt{cycle\_len} parameter? What we've done here is used
a technique called \emph{stochastic gradient descent with restarts
(SGDR)}, a variant of \emph{learning rate annealing}, which gradually
decreases the learning rate as training progresses. This is helpful
because as we get closer to the optimal weights, we want to take smaller
steps.

However, we may find ourselves in a part of the weight space that isn't
very resilient - that is, small changes to the weights may result in big
changes to the loss. We want to encourage our model to find parts of the
weight space that are both accurate and stable. Therefore, from time to
time we increase the learning rate (this is the 'restarts' in 'SGDR'),
which will force the model to jump to a different part of the weight
space if the current area is "spikey". Here's a picture of how that
might look if we reset the learning rates 3 times (in this paper they
call it a "cyclic LR schedule"):

 (From the paper \href{https://arxiv.org/abs/1704.00109}{Snapshot
Ensembles}).

The number of epochs between resetting the learning rate is set by
\texttt{cycle\_len}, and the number of times this happens is refered to
as the \emph{number of cycles}, and is what we're actually passing as
the 2nd parameter to \texttt{fit()}. So here's what our actual learning
rates looked like:

    \begin{Verbatim}[commandchars=\\\{\}]
{\color{incolor}In [{\color{incolor}33}]:} \PY{n}{learn}\PY{o}{.}\PY{n}{sched}\PY{o}{.}\PY{n}{plot\PYZus{}lr}\PY{p}{(}\PY{p}{)}
\end{Verbatim}


    \begin{center}
    \adjustimage{max size={0.9\linewidth}{0.9\paperheight}}{output_76_0.png}
    \end{center}
    { \hspace*{\fill} \\}
    
    Our validation loss isn't improving much, so there's probably no point
further training the last layer on its own.

    Since we've got a pretty good model at this point, we might want to save
it so we can load it again later without training it from scratch.

    \begin{Verbatim}[commandchars=\\\{\}]
{\color{incolor}In [{\color{incolor}15}]:} \PY{n}{learn}\PY{o}{.}\PY{n}{save}\PY{p}{(}\PY{l+s+s1}{\PYZsq{}}\PY{l+s+s1}{224\PYZus{}lastlayer}\PY{l+s+s1}{\PYZsq{}}\PY{p}{)}
\end{Verbatim}


    \begin{Verbatim}[commandchars=\\\{\}]
{\color{incolor}In [{\color{incolor} }]:} \PY{n}{learn}\PY{o}{.}\PY{n}{load}\PY{p}{(}\PY{l+s+s1}{\PYZsq{}}\PY{l+s+s1}{224\PYZus{}lastlayer}\PY{l+s+s1}{\PYZsq{}}\PY{p}{)}
\end{Verbatim}


    \subsubsection{Fine-tuning and differential learning rate
annealing}\label{fine-tuning-and-differential-learning-rate-annealing}

    Now that we have a good final layer trained, we can try fine-tuning the
other layers. To tell the learner that we want to unfreeze the remaining
layers, just call (surprise surprise!) \texttt{unfreeze()}.

    \begin{Verbatim}[commandchars=\\\{\}]
{\color{incolor}In [{\color{incolor}16}]:} \PY{n}{learn}\PY{o}{.}\PY{n}{unfreeze}\PY{p}{(}\PY{p}{)}
\end{Verbatim}


    Note that the other layers have \emph{already} been trained to recognize
imagenet photos (whereas our final layers where randomly initialized),
so we want to be careful of not destroying the carefully tuned weights
that are already there.

Generally speaking, the earlier layers (as we've seen) have more
general-purpose features. Therefore we would expect them to need less
fine-tuning for new datasets. For this reason we will use different
learning rates for different layers: the first few layers will be at
1e-4, the middle layers at 1e-3, and our FC layers we'll leave at 1e-2
as before. We refer to this as \emph{differential learning rates},
although there's no standard name for this techique in the literature
that we're aware of.

    \begin{Verbatim}[commandchars=\\\{\}]
{\color{incolor}In [{\color{incolor}17}]:} \PY{n}{lr}\PY{o}{=}\PY{n}{np}\PY{o}{.}\PY{n}{array}\PY{p}{(}\PY{p}{[}\PY{l+m+mf}{1e\PYZhy{}4}\PY{p}{,}\PY{l+m+mf}{1e\PYZhy{}3}\PY{p}{,}\PY{l+m+mf}{1e\PYZhy{}2}\PY{p}{]}\PY{p}{)}
\end{Verbatim}


    \begin{Verbatim}[commandchars=\\\{\}]
{\color{incolor}In [{\color{incolor}18}]:} \PY{n}{learn}\PY{o}{.}\PY{n}{fit}\PY{p}{(}\PY{n}{lr}\PY{p}{,} \PY{l+m+mi}{3}\PY{p}{,} \PY{n}{cycle\PYZus{}len}\PY{o}{=}\PY{l+m+mi}{1}\PY{p}{,} \PY{n}{cycle\PYZus{}mult}\PY{o}{=}\PY{l+m+mi}{2}\PY{p}{)}
\end{Verbatim}


    
    \begin{verbatim}
HBox(children=(IntProgress(value=0, description='Epoch', max=7), HTML(value='')))
    \end{verbatim}

    
    \begin{Verbatim}[commandchars=\\\{\}]
[ 0.       0.04678  0.02127  0.99219]                         
[ 1.       0.04127  0.01774  0.9917 ]                         
[ 2.       0.03652  0.01744  0.99219]                         
[ 3.      0.0297  0.0206  0.9917]                             
[ 4.       0.0233   0.01944  0.99219]                         
[ 5.       0.01743  0.01844  0.99316]                         
[ 6.       0.02344  0.01892  0.9917 ]                         


    \end{Verbatim}

    Another trick we've used here is adding the \texttt{cycle\_mult}
parameter. Take a look at the following chart, and see if you can figure
out what the parameter is doing:

    \begin{Verbatim}[commandchars=\\\{\}]
{\color{incolor}In [{\color{incolor}23}]:} \PY{n}{learn}\PY{o}{.}\PY{n}{sched}\PY{o}{.}\PY{n}{plot\PYZus{}lr}\PY{p}{(}\PY{p}{)}
\end{Verbatim}


    \begin{center}
    \adjustimage{max size={0.9\linewidth}{0.9\paperheight}}{output_88_0.png}
    \end{center}
    { \hspace*{\fill} \\}
    
    Note that's what being plotted above is the learning rate of the
\emph{final layers}. The learning rates of the earlier layers are fixed
at the same multiples of the final layer rates as we initially requested
(i.e. the first layers have 100x smaller, and middle layers 10x smaller
learning rates, since we set \texttt{lr=np.array({[}1e-4,1e-3,1e-2{]})}.

    \begin{Verbatim}[commandchars=\\\{\}]
{\color{incolor}In [{\color{incolor}19}]:} \PY{n}{learn}\PY{o}{.}\PY{n}{save}\PY{p}{(}\PY{l+s+s1}{\PYZsq{}}\PY{l+s+s1}{224\PYZus{}all}\PY{l+s+s1}{\PYZsq{}}\PY{p}{)}
\end{Verbatim}


    \begin{Verbatim}[commandchars=\\\{\}]
{\color{incolor}In [{\color{incolor}9}]:} \PY{n}{learn}\PY{o}{.}\PY{n}{load}\PY{p}{(}\PY{l+s+s1}{\PYZsq{}}\PY{l+s+s1}{224\PYZus{}all}\PY{l+s+s1}{\PYZsq{}}\PY{p}{)}
\end{Verbatim}


    There is something else we can do with data augmentation: use it at
\emph{inference} time (also known as \emph{test} time). Not
surprisingly, this is known as \emph{test time augmentation}, or just
\emph{TTA}.

TTA simply makes predictions not just on the images in your validation
set, but also makes predictions on a number of randomly augmented
versions of them too (by default, it uses the original image along with
4 randomly augmented versions). It then takes the average prediction
from these images, and uses that. To use TTA on the validation set, we
can use the learner's \texttt{TTA()} method.

    \begin{Verbatim}[commandchars=\\\{\}]
{\color{incolor}In [{\color{incolor}28}]:} \PY{n}{log\PYZus{}preds}\PY{p}{,}\PY{n}{y} \PY{o}{=} \PY{n}{learn}\PY{o}{.}\PY{n}{TTA}\PY{p}{(}\PY{p}{)}
         \PY{n}{probs} \PY{o}{=} \PY{n}{np}\PY{o}{.}\PY{n}{mean}\PY{p}{(}\PY{n}{np}\PY{o}{.}\PY{n}{exp}\PY{p}{(}\PY{n}{log\PYZus{}preds}\PY{p}{)}\PY{p}{,}\PY{l+m+mi}{0}\PY{p}{)}
\end{Verbatim}


    \begin{Verbatim}[commandchars=\\\{\}]
                                                             
    \end{Verbatim}

    \begin{Verbatim}[commandchars=\\\{\}]
{\color{incolor}In [{\color{incolor}20}]:} \PY{n}{accuracy\PYZus{}np}\PY{p}{(}\PY{n}{probs}\PY{p}{,} \PY{n}{y}\PY{p}{)}
\end{Verbatim}


\begin{Verbatim}[commandchars=\\\{\}]
{\color{outcolor}Out[{\color{outcolor}20}]:} 0.991
\end{Verbatim}
            
    I generally see about a 10-20\% reduction in error on this dataset when
using TTA at this point, which is an amazing result for such a quick and
easy technique!

    \subsection{Analyzing results}\label{analyzing-results}

    \subsubsection{Confusion matrix}\label{confusion-matrix}

    \begin{Verbatim}[commandchars=\\\{\}]
{\color{incolor}In [{\color{incolor}30}]:} \PY{n}{preds} \PY{o}{=} \PY{n}{np}\PY{o}{.}\PY{n}{argmax}\PY{p}{(}\PY{n}{probs}\PY{p}{,} \PY{n}{axis}\PY{o}{=}\PY{l+m+mi}{1}\PY{p}{)}
         \PY{n}{probs} \PY{o}{=} \PY{n}{probs}\PY{p}{[}\PY{p}{:}\PY{p}{,}\PY{l+m+mi}{1}\PY{p}{]}
\end{Verbatim}


    A common way to analyze the result of a classification model is to use a
\href{http://www.dataschool.io/simple-guide-to-confusion-matrix-terminology/}{confusion
matrix}. Scikit-learn has a convenient function we can use for this
purpose:

    \begin{Verbatim}[commandchars=\\\{\}]
{\color{incolor}In [{\color{incolor}31}]:} \PY{k+kn}{from} \PY{n+nn}{sklearn}\PY{n+nn}{.}\PY{n+nn}{metrics} \PY{k}{import} \PY{n}{confusion\PYZus{}matrix}
         \PY{n}{cm} \PY{o}{=} \PY{n}{confusion\PYZus{}matrix}\PY{p}{(}\PY{n}{y}\PY{p}{,} \PY{n}{preds}\PY{p}{)}
\end{Verbatim}


    We can just print out the confusion matrix, or we can show a graphical
view (which is mainly useful for dependents with a larger number of
categories).

    \begin{Verbatim}[commandchars=\\\{\}]
{\color{incolor}In [{\color{incolor}32}]:} \PY{n}{plot\PYZus{}confusion\PYZus{}matrix}\PY{p}{(}\PY{n}{cm}\PY{p}{,} \PY{n}{data}\PY{o}{.}\PY{n}{classes}\PY{p}{)}
\end{Verbatim}


    \begin{Verbatim}[commandchars=\\\{\}]
[[996   4]
 [  8 992]]

    \end{Verbatim}

    \begin{center}
    \adjustimage{max size={0.9\linewidth}{0.9\paperheight}}{output_102_1.png}
    \end{center}
    { \hspace*{\fill} \\}
    
    \subsubsection{Looking at pictures
again}\label{looking-at-pictures-again}

    \begin{Verbatim}[commandchars=\\\{\}]
{\color{incolor}In [{\color{incolor}38}]:} \PY{n}{plot\PYZus{}val\PYZus{}with\PYZus{}title}\PY{p}{(}\PY{n}{most\PYZus{}by\PYZus{}correct}\PY{p}{(}\PY{l+m+mi}{0}\PY{p}{,} \PY{k+kc}{False}\PY{p}{)}\PY{p}{,} \PY{l+s+s2}{\PYZdq{}}\PY{l+s+s2}{Most incorrect cats}\PY{l+s+s2}{\PYZdq{}}\PY{p}{)}
\end{Verbatim}


    \begin{Verbatim}[commandchars=\\\{\}]
Most incorrect cats

    \end{Verbatim}

    \begin{center}
    \adjustimage{max size={0.9\linewidth}{0.9\paperheight}}{output_104_1.png}
    \end{center}
    { \hspace*{\fill} \\}
    
    \begin{Verbatim}[commandchars=\\\{\}]
{\color{incolor}In [{\color{incolor}39}]:} \PY{n}{plot\PYZus{}val\PYZus{}with\PYZus{}title}\PY{p}{(}\PY{n}{most\PYZus{}by\PYZus{}correct}\PY{p}{(}\PY{l+m+mi}{1}\PY{p}{,} \PY{k+kc}{False}\PY{p}{)}\PY{p}{,} \PY{l+s+s2}{\PYZdq{}}\PY{l+s+s2}{Most incorrect dogs}\PY{l+s+s2}{\PYZdq{}}\PY{p}{)}
\end{Verbatim}


    \begin{Verbatim}[commandchars=\\\{\}]
Most incorrect dogs

    \end{Verbatim}

    \begin{center}
    \adjustimage{max size={0.9\linewidth}{0.9\paperheight}}{output_105_1.png}
    \end{center}
    { \hspace*{\fill} \\}
    
    \subsection{Review: easy steps to train a world-class image
classifier}\label{review-easy-steps-to-train-a-world-class-image-classifier}

    \begin{enumerate}
\def\labelenumi{\arabic{enumi}.}
\tightlist
\item
  Enable data augmentation, and precompute=True
\item
  Use \texttt{lr\_find()} to find highest learning rate where loss is
  still clearly improving
\item
  Train last layer from precomputed activations for 1-2 epochs
\item
  Train last layer with data augmentation (i.e. precompute=False) for
  2-3 epochs with cycle\_len=1
\item
  Unfreeze all layers
\item
  Set earlier layers to 3x-10x lower learning rate than next higher
  layer
\item
  Use \texttt{lr\_find()} again
\item
  Train full network with cycle\_mult=2 until over-fitting
\end{enumerate}

    \subsection{Understanding the code for our first
model}\label{understanding-the-code-for-our-first-model}

    Let's look at the Dogs v Cats code line by line.

\textbf{tfms} stands for \emph{transformations}.
\texttt{tfms\_from\_model} takes care of resizing, image cropping,
initial normalization (creating data with (mean,stdev) of (0,1)), and
more.

    \begin{Verbatim}[commandchars=\\\{\}]
{\color{incolor}In [{\color{incolor}29}]:} \PY{n}{tfms} \PY{o}{=} \PY{n}{tfms\PYZus{}from\PYZus{}model}\PY{p}{(}\PY{n}{resnet34}\PY{p}{,} \PY{n}{sz}\PY{p}{)}
\end{Verbatim}


    We need a path that points to the dataset. In this path we will also
store temporary data and final results.
\texttt{ImageClassifierData.from\_paths} reads data from a provided path
and creates a dataset ready for training.

    \begin{Verbatim}[commandchars=\\\{\}]
{\color{incolor}In [{\color{incolor}30}]:} \PY{n}{data} \PY{o}{=} \PY{n}{ImageClassifierData}\PY{o}{.}\PY{n}{from\PYZus{}paths}\PY{p}{(}\PY{n}{PATH}\PY{p}{,} \PY{n}{tfms}\PY{o}{=}\PY{n}{tfms}\PY{p}{)}
\end{Verbatim}


    \texttt{ConvLearner.pretrained} builds \emph{learner} that contains a
pre-trained model. The last layer of the model needs to be replaced with
the layer of the right dimensions. The pretained model was trained for
1000 classes therfore the final layer predicts a vector of 1000
probabilities. The model for cats and dogs needs to output a two
dimensional vector. The diagram below shows in an example how this was
done in one of the earliest successful CNNs. The layer "FC8" here would
get replaced with a new layer with 2 outputs.


\href{https://image.slidesharecdn.com/practicaldeeplearning-160329181459/95/practical-deep-learning-16-638.jpg}{original
image}

    \begin{Verbatim}[commandchars=\\\{\}]
{\color{incolor}In [{\color{incolor}31}]:} \PY{n}{learn} \PY{o}{=} \PY{n}{ConvLearner}\PY{o}{.}\PY{n}{pretrained}\PY{p}{(}\PY{n}{resnet34}\PY{p}{,} \PY{n}{data}\PY{p}{,} \PY{n}{precompute}\PY{o}{=}\PY{k+kc}{True}\PY{p}{)}
\end{Verbatim}


    \emph{Parameters} are learned by fitting a model to the data.
\emph{Hyparameters} are another kind of parameter, that cannot be
directly learned from the regular training process. These parameters
express ``higher-level'' properties of the model such as its complexity
or how fast it should learn. Two examples of hyperparameters are the
\emph{learning rate} and the \emph{number of epochs}.

During iterative training of a neural network, a \emph{batch} or
\emph{mini-batch} is a subset of training samples used in one iteration
of Stochastic Gradient Descent (SGD). An \emph{epoch} is a single pass
through the entire training set which consists of multiple iterations of
SGD.

We can now \emph{fit} the model; that is, use \emph{gradient descent} to
find the best parameters for the fully connected layer we added, that
can separate cat pictures from dog pictures. We need to pass two
hyperameters: the \emph{learning rate} (generally 1e-2 or 1e-3 is a good
starting point, we'll look more at this next) and the \emph{number of
epochs} (you can pass in a higher number and just stop training when you
see it's no longer improving, then re-run it with the number of epochs
you found works well.)

    \begin{Verbatim}[commandchars=\\\{\}]
{\color{incolor}In [{\color{incolor}32}]:} \PY{n}{learn}\PY{o}{.}\PY{n}{fit}\PY{p}{(}\PY{l+m+mf}{1e\PYZhy{}2}\PY{p}{,} \PY{l+m+mi}{1}\PY{p}{)}
\end{Verbatim}


    
    \begin{verbatim}
A Jupyter Widget
    \end{verbatim}

    
    \begin{Verbatim}[commandchars=\\\{\}]
[ 0.       0.04153  0.02681  0.98877]                          


    \end{Verbatim}

    \subsection{Analyzing results: loss and
accuracy}\label{analyzing-results-loss-and-accuracy}

    When we run \texttt{learn.fit} we print 3 performance values (see
above.) Here 0.03 is the value of the \textbf{loss} in the training set,
0.0226 is the value of the loss in the validation set and 0.9927 is the
validation accuracy. What is the loss? What is accuracy? Why not to just
show accuracy?

\textbf{Accuracy} is the ratio of correct prediction to the total number
of predictions.

In machine learning the \textbf{loss} function or cost function is
representing the price paid for inaccuracy of predictions.

The loss associated with one example in binary classification is given
by: \texttt{-(y\ *\ log(p)\ +\ (1-y)\ *\ log\ (1-p))} where \texttt{y}
is the true label of \texttt{x} and \texttt{p} is the probability
predicted by our model that the label is 1.

    \begin{Verbatim}[commandchars=\\\{\}]
{\color{incolor}In [{\color{incolor}34}]:} \PY{k}{def} \PY{n+nf}{binary\PYZus{}loss}\PY{p}{(}\PY{n}{y}\PY{p}{,} \PY{n}{p}\PY{p}{)}\PY{p}{:}
             \PY{k}{return} \PY{n}{np}\PY{o}{.}\PY{n}{mean}\PY{p}{(}\PY{o}{\PYZhy{}}\PY{p}{(}\PY{n}{y} \PY{o}{*} \PY{n}{np}\PY{o}{.}\PY{n}{log}\PY{p}{(}\PY{n}{p}\PY{p}{)} \PY{o}{+} \PY{p}{(}\PY{l+m+mi}{1}\PY{o}{\PYZhy{}}\PY{n}{y}\PY{p}{)}\PY{o}{*}\PY{n}{np}\PY{o}{.}\PY{n}{log}\PY{p}{(}\PY{l+m+mi}{1}\PY{o}{\PYZhy{}}\PY{n}{p}\PY{p}{)}\PY{p}{)}\PY{p}{)}
\end{Verbatim}


    \begin{Verbatim}[commandchars=\\\{\}]
{\color{incolor}In [{\color{incolor}35}]:} \PY{n}{acts} \PY{o}{=} \PY{n}{np}\PY{o}{.}\PY{n}{array}\PY{p}{(}\PY{p}{[}\PY{l+m+mi}{1}\PY{p}{,} \PY{l+m+mi}{0}\PY{p}{,} \PY{l+m+mi}{0}\PY{p}{,} \PY{l+m+mi}{1}\PY{p}{]}\PY{p}{)}
         \PY{n}{preds} \PY{o}{=} \PY{n}{np}\PY{o}{.}\PY{n}{array}\PY{p}{(}\PY{p}{[}\PY{l+m+mf}{0.9}\PY{p}{,} \PY{l+m+mf}{0.1}\PY{p}{,} \PY{l+m+mf}{0.2}\PY{p}{,} \PY{l+m+mf}{0.8}\PY{p}{]}\PY{p}{)}
         \PY{n}{binary\PYZus{}loss}\PY{p}{(}\PY{n}{acts}\PY{p}{,} \PY{n}{preds}\PY{p}{)}
\end{Verbatim}


\begin{Verbatim}[commandchars=\\\{\}]
{\color{outcolor}Out[{\color{outcolor}35}]:} 0.164252033486018
\end{Verbatim}
            
    Note that in our toy example above our accuracy is 100\% and our loss is
0.16. Compare that to a loss of 0.03 that we are getting while
predicting cats and dogs. Exercise: play with \texttt{preds} to get a
lower loss for this example.

\textbf{Example:} Here is an example on how to compute the loss for one
example of binary classification problem. Suppose for an image x with
label 1 and your model gives it a prediction of 0.9. For this case the
loss should be small because our model is predicting a label \(1\) with
high probability.

\texttt{loss\ =\ -log(0.9)\ =\ 0.10}

Now suppose x has label 0 but our model is predicting 0.9. In this case
our loss should be much larger.

loss = -log(1-0.9) = 2.30

\begin{itemize}
\tightlist
\item
  Exercise: look at the other cases and convince yourself that this make
  sense.
\item
  Exercise: how would you rewrite \texttt{binary\_loss} using
  \texttt{if} instead of \texttt{*} and \texttt{+}?
\end{itemize}

Why not just maximize accuracy? The binary classification loss is an
easier function to optimize.


    % Add a bibliography block to the postdoc
    
    
    
    \end{document}
